\section{Introduction}
本章给出了在相对论性量子理论中的\C \P \T 定义。
包括场的量子化等
\section{记号}
在这个注中,我使用与书中不同的记号来写注记

\begin{equation}
    c = \hbar = 1
\end{equation}

\begin{equation}
    \cocovector g \mu \nu = \mathrm{diag}[-1, 1, 1, 1]
\end{equation}

坐标
\begin{equation}
    \bivector x \mu = (t, \vec x)
\end{equation}

\begin{equation}
    \covector x \mu = (-t, \vec x)
\end{equation}

能动量
\begin{equation}
    \bivector p \mu = (E, \vec p)
\end{equation}

\begin{equation}
    \covector p \mu = (-E, \vec p)
\end{equation}

\begin{equation}
    \bivector \partial \mu = \frac \partial {\partial \covector x \mu}
\end{equation}

\begin{equation}
    \covector \partial \mu = \frac \partial {\partial \bivector x \mu}
\end{equation}

\begin{equation}
    \bivector p \mu = -i \bivector \partial \mu
\end{equation}

Dirac matrices

\diracgamma 5 = i \diracgamma 0 \diracgamma 1 \diracgamma 2 \diracgamma 3 \diracgamma 4

%\section{自由场}
\section{Spin-0场的量子化及其\C \P \T}
\subsection{实标量场量子化}
为了讨论复标量场,不妨先从实标量场入手

把Klein-Gordon场方程的解展开
\begin{equation}
    \phi(t, \vec x) = \int \VolumeUnit k \omega \left[ a(k) e^{i\vec k \cdot \vec x - \i\omega_k t} + a^*(k)e^{-i\vec k \cdot \vec x + i \omega_k t} \right]
    \equiv \int d^3\vec k = \left[a(k)f_k(x) + a^*(k)f_k^*(x)\right]
\end{equation}
量子化过程即将上式写成算符
\begin{equation}
    \phi(t, \vec x) = \int \VolumeUnit k \omega \left[ a(k) e^{i\vec k \cdot \vec x - \i\omega_k t} + a^\dagger(k)e^{-i\vec k \cdot \vec x + i \omega_k t} \right]
    \equiv \int d^3\vec k = \left[a(k)f_k(x) + a^\dagger(k)f_k^\dagger(x)\right]
\end{equation}
同时,正则动量可以求出
\begin{equation}
    \pi(t, \vec x) = \dot \phi(t, x) = \int \VolumeUnit k \omega \left[ i\omega_k a(k) e^{i\vec k \cdot \vec x - \i\omega_k t} - i\omega_k a^\dagger(k)e^{-i\vec k \cdot \vec x + i \omega_k t} \right]
    \equiv \int d^3\vec k = \left[a(k)f_k(x) + a^\dagger(k)f_k^\dagger(x)\right]
\end{equation}

可以推出
\begin{equation}
    a(k) = i \int d^3 \vec x f_k^*\partial_0\phi
\end{equation}
\begin{equation}
    a^\dagger(k) = -i \int d^3 \vec x f_k\partial_0\phi
\end{equation}
为了保持洛仑兹协变性
\begin{equation}
\ket {\vec k} = \sqrt{(2\pi)^32E_{\vec k)}} a^\dagger(\vec k)\ket{0}
\end{equation}

\subsection{复标量场量子化}
讨论量子化的复标量场
\begin{equation}
    \phi(t, x) = \frac 1  2 [\phi_1(t, x) + i\phi_2(t, x)]
\end{equation}

可以量子化为
\begin{equation}
    \phi(x) = \int \VolumeUnit k \omega \left[a_+(k)e\right]
\end{equation}

\subsection{标量场的\C \P \T}
\subsubsection{\C 变换}
将复标量场的\C 变换定义为
\begin{equation}
    \UC\phi(x)\invUC = \eta_C\phi^\dagger(x)
\end{equation}
取复共轭可得
\begin{equation}
    \UC a_{vec k}\invUC = \eta_C b_{\vec k}
\end{equation}
带入$\phi$的表达式可得,
\begin{equation}
    \begin{array}{ccc}
        \UC a_{\vec k} \invUC &=& \eta_C b_{\vec k} \\
        \UC a_{\vec k}^\dagger \invUC &=& \eta_C^* b_{\vec k}^\dagger \\
        \UC b_{\vec k} \invUC &=& \eta_C^* a_{\vec k} \\
        \UC b_{\vec k}\dagger \invUC &=& \eta_C a_{\vec k}^\dagger \\
\end{array}
\end{equation}
对于实标量场,可得
\begin{equation}
    \eta_C = \pm 1
\end{equation}
\subsubsection{\P 变换}
\P 变换表示将坐标变换为
\begin{equation}
    \vec x \rightarrow \vec x^\prime = -\vec x
\end{equation}
在\P 变换下,复标量场的变化就应该是
\begin{equation}
    \UP\phi(t,\vec x)\invUP = \eta_P\phi(t,-\vec x)
    \end{equation}
\begin{equation}
    \UP\phi\dagger(t,\vec x)\invUP = \eta_P^*\phi\dagger(t,-\vec x)
    \end{equation}
    同样,对于实标量场

\begin{comment}
\section{Spin-1场的量子化及其\C \P \T}
Maxwell 方程组可以写作
\begin{equation}
    \covector \partial \mu \bibivector F \mu \v = e \bivector J \nu
\end{equation}
其中
\begin{equation}
    \bibivector F \mu \nu = \FuvExpand
    \bibivector F \mu \nu \equiv \bivector \partial \mu \bivector A \nu 
    - \bivector \partial \nu \bivector A \mu
\end{equation}

\begin{equation}
    \L = \EMLagrangianExpand
\end{equation}
由于规范问题,电动力学不能直接量子化,采用库仑规范
\begin{equation}
    \CoulombGaugeCondition
\end{equation}

那么,$\vec A$的正则动量就是:
\begin{equation}
    \vec \Pi(t, \vec x) = 
    -\frac \partial {\partial t} \vec A (t, \vec x) - \vec \nabla A_0 = \vec E
\end{equation}

\end{comment}
